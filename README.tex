\documentclass[11pt]{article}
\usepackage{physics}

\begin{document}
This repository contains training data for an ML-based parameter optimization for modeling morphogen gradient formation by a source-diffusion-sink (SDS) mechanism, as described by the following one-dimensional (1D) reaction-diffusion partial differential equation:

\begin{equation}\label{eq:SDS_1D_PDE}
	\frac{\partial c(x,t)}{\partial t} = f_{\rm{source}}(x) + D\frac{\partial^2 }{\partial x^2}c(x,t) - k_{\rm{sink}}c(x,t)\,.
\end{equation}

Here, $c(x,t)$ is the scalar concentration field of the morphogen in space $x$ at time $t$, $D$ is the constant homogeneous diffusion coefficient, $k_{\rm{sink}}$ is the sink rate scaling with $c(x,t)$, describing morphogen degradation by the cells and proteases in the extracellular space, and $f_{\rm{source}}$ is the source term describing morphogen secretion by the source cells. 

$f_{\rm{source}}$ depends on the location as only a group of cells produce the morphogen, i.e.,
\begin{equation}\label{eq:source}
	f_{\rm{source}}(x) =
        \begin{cases}
        k_{\rm{source}} \quad & \text{for} \quad  0 \le x \le w_{\rm{source}}  \\
        0 & \rm{otherwise}\, ,
    \end{cases}  
\end{equation}
%
with source width $w_{\rm{source}} = 0.3 L$ in a 1D diffusion domain of length $L$. 

Solving Eq.~\eqref{eq:SDS_1D_PDE} until steady-state results in a morphogen concentration profile that can flat, exponential, or step-wise depending on the parameters $k_{\rm{source}}$, $k_{\rm{sink}}$, and $D$. The goal is to generate a model that predicts a set of parameters given an input gradient. To train this model, we produce simulated training data by solving Eq.~\eqref{eq:SDS_1D_PDE} until steady-state for different parameter sets.

The output folders contain a parameters.csv file, where the first row defines the parameter type of the respective column, and each row below contains a set of parameters with index $i$. The corresponding gradients folder contains gradient\_$i$.csv files, each corresponding to one row $i$ of the parameters.csv file. 

Each gradient\_$i$.csv file contains two columns: the first containing $x$, the second containing the steady-state concentration field $c(x, t_{\rm{max}})$.
\end{document}
